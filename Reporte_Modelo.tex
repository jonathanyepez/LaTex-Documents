\documentclass[11pt,a4paper,roman]{article}
\usepackage{geometry}
\usepackage{graphicx}
\usepackage{fancyhdr}
\usepackage{tabto} 
\usepackage{datetime}
\usepackage{titlepic}
\usepackage{array}
\usepackage{multirow}

%\pagestyle{empty}
%\geometry{headheight=0.75in}
\geometry{margin = 0.75in}
%\TabPositions{0pt,0.35,0.7\linewidth}

% Title Page
\title{\textbf{Plan de Ejecución del Proyecto}\\[0.25in] Ampliación Mr. Joy "El Recreo"\\[0.5in]}

\titlepic{\includegraphics[width=0.5\linewidth]{portadaMrJoy}}
%\begin{figure}
%	\centering
%	\includegraphics[width=0.5\linewidth]{portadaMrJoy}
%\end{figure}
\author{Franshesca Yomara Yépez Andino\\[0.5in]}
\date{Miércoles, 3 de Julio de 2019\\[0.5in]}

\renewcommand{\contentsname}{} %gets rid of the 'Contents Title'
\renewcommand\tablename{Cuadro}

\begin{document}
\maketitle
\thispagestyle{empty}

\newpage

%	\begin{titlepage}
%		\centering
%		\vfill
%		{\bfseries\Large
%			Performance Report\\
%			First Quarter 2011\\
%			\vskip2cm
%			\includegraphics[width=4cm]{portadaMrJoy}\\ % also works with logo.pdf
%			A. Uthor\\
%		}    
%		\vfill
%		
%		\vfill
%		\vfill
%	\end{titlepage}

\textbf{\begin{center}
		{\Large Índice General}
\end{center}}
\tableofcontents
\thispagestyle{empty}


\newpage
\section{Ámbito General del Proyecto}
El propósito de este documento es presentar un resumen ejecutivo del ámbito general del proyecto: AMPLIACIÓN Mr. JOY EL RECREO en el cual se establecen las bases al Plan de Gestión (PGP) o Ejecución (PEP) del mismo.\\
\subsection{Identificación del Proyecto}
El cuadro mostrado a continuación presenta información general que logra identificar al proyecto y su contexto.
\begin{table}[h]
\caption{Aspectos de Identificación del Negocio}
\label{tab:identneg}
\begin{flushleft}
\begin{tabular}{|p{3cm}|p{4cm}|p{3cm}|p{3cm}|}
			\hline \textbf{Nombre del Proyecto:}
			& \multicolumn{3}{c|}{\textbf{Ingeniería, Procura y Construcción de la Ampliación de Mr.Joy El Recreo}} \\ 
			\hline \textbf{Seudónimo:}
			& Ampliación Mr. Joy El Recreo & \textbf{Código o ID:} & 000 \\ 
			\hline \textbf{Patrocinador:}
			& Active Fun Diversion S.A. & \textbf{Representante:} & Peter Taylor \\ 
			\hline \textbf{Unidad Ejecutora:}
			& Active Fun Diversion S.A. & \textbf{Representante:} & Peter Taylor \\ 
			\hline \textbf{Ubicación:}
			& \multicolumn{3}{c|}{Av. Pedro Maldonado C.C El Recreo.} \\ 
			\hline 
\end{tabular}
\end{flushleft}
\end{table}
\subsection{Justificación del Proyecto}
La AMPLIACIÓN Mr. JOY EL RECREO se considera principalmente, para darle un nuevo aire entre innovación y remodelación al primer parque activo desde hace unos 6 años.\\[0.25in]
Esta inversión presume un atractivo para quienes se esperan sean  los principales usuarios del parque, permitiendo obtener un resultado ganar-ganar tanto para el socio/cliente que incrementa sus ventas, como para el usuario que percibirá una ganancia en diversión con las mejoras de los juegos y atractivos.\\[0.25in]
En el área financiera el comportamiento económico no tenía cambios muy relevantes, evidenciándose un aumento considerable posterior a la apertura, lográndose de esta manera el objetivo.\\[0.25in]

\begin{figure}[h]
	\centering
	\includegraphics[width=0.8\linewidth]{chart1}
	\caption[chart1]{chart1}
	\label{fig:chart1}
\end{figure}

\noindent
Luego de la implementación de la remodelación a finales del mes de noviembre 2018, los ingresos del centro muestran una mejora significativa, comparando con periodos anteriores. Tal como se muestra en el grafico anterior. Es muy notorio los resultados de los últimos dos meses, donde nuestra variación \% promedio es de 19.5\% y si tomamos en cuenta que en los 5 meses antes de la remodelación manteníamos una variación negativa promedio de -17.4\%, estamos hablando de un resultado positivo incidido por este proyecto de remodelación de hasta un 35\% de efecto positivo en nuestros ingresos.\\[0.25in]

\subsection{Necesidades y Objetivos del Negocio}
En siguiente cuadro se exponen los objetivos del negocio (Distintos a los objetivos del proyecto) implícitos al proyecto y la organización, que para este caso en particular atienden a una necesidad del mercado.
\begin{table}[h]
\caption{Objetivos del Negocio}
\label{tab:objetivosneg}	

\begin{tabular}{|p{1cm}|p{14cm}|}
	\hline \textbf{Nro.}
	& \textbf{Objetivos del Negocio} \\ 
	\hline \textbf{01}
	& Mejorar el perfil de ventas en relación a períodos anteriores \\ 
	\hline \textbf{02}
	& Ofrecer novedad a los clientes de la parte Sur de la Ciudad, marcando una pauta frente a la competencia. \\ 
	\hline \textbf{03}
	& Generar fuentes de empleo. \\ 
	\hline \textbf{04}
	& Mantener la sostenibilidad del negocio, siendo éste el primer Parque Mr. Joy en el Ecuador, dándole un nuevo aire con mejores y nuevos juegos. \\ 
	\hline 
\end{tabular} 
\end{table}
\subsection{Descripción General del Proyecto}
El proyecto básicamente consiste en la ejecución de la Ingeniería, Procura y Construcción de la Ampliación Mr. Joy el Recreo. Se constituye por las tres fases mencionadas, las cuales interactúan de forma secuencial y simultánea durante el transcurso del ciclo de vida del proyecto.\\[0.25in]
La primera fase constituye la Ingeniería; conformada por una serie de procesos de  Evaluación y Diseño que darán como resultados todos los documentos y planos entregables necesarios, previo a la operatividad. Esta fase fijará la complejidad y formalidad del proyecto, al considerar todos los estudios ineludibles hasta alcanzar la ingeniería de detalle.\\[0.25in] 
La segunda fase representa la Procura; conformada por una serie de procesos previos a la Implementación cuyo fin es alcanzar la requisición y adquisición  formal de los materiales, equipos, herramientas y servicios especiales necesarios, derivados del presupuesto. En esta fase serán recopilados aquellos elementos indispensables para elaborar un sistema constructivo que garantice la durabilidad de la obra, la seguridad plena de los usuarios, y una alternativa económica fiable.\\[0.25in]
La tercera y última fase corresponde a la Construcción y/o Implementación; conformada por una serie de procesos  específicos cuyo objetivo es cumplir con la entrega del producto final y operatividad definitiva, en este caso, la Ampliación de Mr. Joy el Recreo. Esta fase básicamente pretende materializar todos los objetos físicos esbozados en el proyecto, incluyendo la elaboración de la edificación estructural en la carpa de Joy Laser y Trampolines, la dotación de los servicios eléctricos, sanitarios, incendio, nuevas adecuaciones de juegos, y los nuevos mobiliarios, cambio de imagen publicitario, etc.\\

\subsection{Premisas e Hitos Principales}
Las premisas o condiciones principales y sobre las cuales se desarrolla el acta de constitución, son:

\begin{itemize}
	\item Los entregables obtenidos durante la fase de Ingeniería, resultan de los estudios de diseño y espacio adecuando las áreas con los servicios para la ubicación de los juegos del parque.
	\item Los recursos materiales para la construcción adquiridos durante la fase de procura, y control de calidad.
	\item Toda la permisería ante el centro comercial será solicitada con la suficiente antelación.
	\item Los detalles y acabados arquitectónicos son rigurosamente evaluados por el Project Manager, para garantizar el cumplimiento de los estándares exigidos. De ser así, los entregables parciales en la fase de construcción, pertenecientes a la disciplina de arquitectura, serán señalados como válidos.
	\item Realización de pruebas, hidrostáticas, eléctricas, de compresión del concreto u otras pruebas aplicadas sobre los diferentes sistemas de servicios y sus componentes, estarán respaldados y aceptados por el Project Manager.
	\item El periodo máximo establecido para la implementación  que pertenece al ciclo de vida del proyecto, es de un (1) mes  y cuatro (4) días. Tomando como guía el calendario laboral.
\end{itemize}

\noindent
El cuadro expuesto a continuación, identifica todos los hitos principales y obligatorios necesarios para cumplir con los objetivos y obligaciones contractuales del proyecto.

\begin{table}[h]
	
\caption{Hitos y Fechas}
\label{tab:cuadro3}	

\begin{tabular}{|p{6cm}|p{6cm}|p{3cm}|}
	\hline \textbf{\textit{Fase}} & \textbf{\textit{Hitos}} & \textbf{\textit{Fecha}}\\
	\hline \textbf{Ingeniería/Diseño} & Aprobación de Planos & 22/10/2018\\
	\hline 
	\multirow{3}{*}{\textbf{Procura}} & {Firma de contrato con proveedores} & {24/10/2018} \\\cline{2-3}
	& {Adquisición de los recursos materiales} & {24/10/2018} \\\cline{2-3}
	& {Obtención de permiso por parte del CC el Recreo} & {22/10/2018} \\\cline{2-3}
	\hline
	\multirow{5}{*}{\textbf{Construcción/Implementación}} & {Obras Civiles} & {22/10/2018} \\\cline{2-3}
	& {Demolición de pared de galpón} & {18/11/2018} \\\cline{2-3}
	& {Obras eléctricas, CCTV} & {22/10/2018} \\\cline{2-3}
	&{Instalación de Juegos} & {19/11/2018} \\\cline{2-3}
	&{Evaluación de calidad sobre acabados arquitectónicos y constructivos.} & {02/11/2018} \\\hline
\end{tabular}

\end{table}

\end{document}          
